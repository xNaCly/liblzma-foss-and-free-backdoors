\section{Introduction}

FOSS\footnote{Free and Open-Source Software \cite{stallmann2021copyleft}} is
generally defined as software the user can ``[...] run, copy, distribute,
study, change and improve [...]'' \cite{fsf2024whycopylef}. This requires
the source to be available and enables the dependence of other software on
subsets or the entirety of the code. On the other hand, source available or
OSS\footnote{Open-Source Software} are distinct from FOSS software. Some
licenses do not require the resulting product to be licensed under the same
license as its dependencies, such as the MIT license\footnote{Requires the
license to be present in ``all copies or substantial portions of the
Software'' \cite{osorg2024mit}}. It therefore differs from the
GPL\footnote{Requires all copies of the software to be licensed as GPL
\cite{osorg2024gpl}} and software licensed with the MIT-Licence can
therefore not be referred to as free open-source software, but rather as
open-source software. \newline 
Most OSS-projects accept contributions from individuals and enterprises.
This is wanted and required to support the actuality of said software. Most
OSS projects accept changes matching their pre-defined contribution
guidelines and credit the contributor for their addition. These contributors
often use the software they are contributing to and therefore make changes
they care for, such as adding drivers for new devices to the linux kernel
\cite{linuxUnknownDevicedrivers}. However, other independent OSS
contributors are abusing the contribution system by exploiting the trust the
unpaid maintainers have in the quality of the submitted changes. As was the
case with \texttt{liblzma} or the \texttt{xz-utils} OSS library.

\subsection{Dependence on FOSS and OSS}

Open source software is often divided into reusable components, such as
libraries or toolkits implementing a specific feature, and built upon by other
software. The goal is to use tried and tested components in the creation of new
OSS, thus building on field tested and established software found in the OSS
community.

Not only does OSS depend on other libraries from the OSS ecosystem. Proprietary
software also makes use of said OSS components, while being forced to adhere to
their terms, as declared in their respective licenses \cite{libcurl2024usage}. 


\begin{figure}[H]
    \includegraphics[scale=0.5]{assets/dependency.png}
    \caption{Dependency \cite{xkcdUnknownDependency}}
    \label{img:dependency}
\end{figure}

Commonly used examples for said libraries are \texttt{libcurl}, which provides
multiprotocol file transfers \cite{libcurl2024overview}, \texttt{raylib} which
is a libary for videogames programming \cite{raylib2024landing} and the
\texttt{sqlite} library, that implements a in process database
\cite{sqlite2024landing}.

\subsection{Supply Chain Security}

\cite{ccc2024supplychainsec}

\subsection{xz-utils and liblzma}

\cite{ccc2024backdoor}
