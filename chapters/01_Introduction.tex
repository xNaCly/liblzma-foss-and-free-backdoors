\section{Introduction}

FOSS\footnote{Free and Open-Source Software \cite{stallmann2021copyleft}} is
generally defined as software the user can ``[...] run, copy, distribute,
study, change and improve [...]'' \cite{fsf2024whycopylef}. This requires the
source to be available and enables the dependence of other software on subsets
or the entirety of the code. On the other hand, source available or
OSS\footnote{Open-Source Software} are distinct from FOSS software. Some
licenses do not require the resulting product to be licensed under the same
license as its dependencies, such as the MIT license\footnote{Requires the
license to be present in ``all copies or substantial portions of the Software''
\cite{osorg2024mit}}. It therefore differs from the GPL\footnote{Requires all
copies of the software to be licensed as GPL \cite{osorg2024gpl}} and software
licensed with the MIT-Licence can therefore not be referred to as free
open-source software, but rather as open-source software.

\subsection{Supply Chain Security}
\subsection{Dependence on FOSS and OSS}
\subsection{xz-utils and liblzma}
