\section{Introduction}

FOSS\footnote{Free and Open-Source Software \cite{stallmann2021copyleft}} is
generally defined as software the user can ``[...] run, copy, distribute,
study, change and improve [...]'' \cite{fsf2024whycopylef}. This requires the
source to be available and enables the dependence of other software on subsets
or the entirety of the code. On the other hand, source available or
OSS\footnote{Open-Source Software} are distinct from FOSS software. Some
licenses do not require the resulting product to be licensed under the same
license as its dependencies, such as the MIT license\footnote{Requires the
license to be present in ``all copies or substantial portions of the Software''
\cite{osorg2024mit}}. It therefore differs from the GPL\footnote{Requires all
copies of the software to be licensed as GPL \cite{osorg2024gpl}} and software
licensed with the MIT-Licence can therefore not be referred to as free
open-source software, but rather as open-source software.

% TODO: WRITE MORE HERE


\subsection{Dependence on FOSS and OSS}

Open source software is often divided into reusable components, such as
libraries or toolkits implementing a specific feature, and built upon by other
software. The goal is to use tried and tested components in the creation of new
OSS, thus building on field tested and established software found in the OSS
community.

Not only does OSS depend on other libraries from the OSS ecosystem. Proprietary
software also makes use of said OSS components, while being forced to adhere to
their terms, as declared in their respective licenses.

\subsection{Supply Chain Security}

\cite{ccc2024supplychainsec}

\subsection{xz-utils and liblzma}

\cite{ccc2024backdoor}
