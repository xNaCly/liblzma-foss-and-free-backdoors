\section{Conclusion}

In conclusion the \texttt{xz-utils}/\texttt{libzlma} backdoor is the result of
years of work by a highly sophisticated threat actor. The backdoor
implementation shows the deep knowledge of low level programming, obscure
compiler features, execution formats, the open source software supply chain and
its security while exhibiting the ability for total control over the threat
actors operational security. No group or individual have taken credit or were
accredited to have engineered this backdoor.

\subsection{Funding FOSS and OSS}

To fund open source software projects, means to provide the maintainers of
said project to spend more time on bug reports, feature requests and
securing the project itself. Often, funding enables the maintainers to work
on the software full-time and thus allow a stable financial situation which
could prevent issues such as the \texttt{xz-utils} backdoor in the future.

\subsection{Vetting Dependency}

As shown in \autoref{sec:dependence}, not only the direct dependencies of a
given software project have to be secure, but their dependencies too. To ensure
this, a large effort has to be made to vet this source code.

Making sure source code is secure is a combination of static and dynamic
analysis, testing, fuzzing and human analysis. Ignoring the last method, all
can be largely automated, human analysis however is a time consuming process
and requires manual labor.
