\section{Response}

On Match 29th, 2024 \textit{Andres Freund}, a PostgreSQL developer at
Microsoft, discovered CPU usage spikes and Valgrind errors while attempting
to establish SSH connections from Debian sid\footnote{Debian Unstable (Sid)
refers to rolling development version of Debian} using \texttt{openssh}. He
assesses the \textit{libzlma} distributions \texttt{v5.6.0} and
\texttt{v5.6.1} to contain the backdoor. According to Freund, the runtime
impact on an infected system is substantial, specifically logging in via ssh
introduces a slowdown of around 500ms. \cite{openwall2024backdoor}

\subsection{Software Distributors Reactions}

After Freud's discovery, Red Hat, SUSE and Debian downgraded their shipped
\texttt{libzlma} dependency from the affected versions to previous
distributions, see \cite{redhat2024xz}, \cite{debian2024xz} and
\cite{suse2024xz}. Ubuntu held back its development release, Ubuntu Noble
24.04, for a deep-dive and removed the affected library (\texttt{liblzma})
\cite{ubuntu2024xz}.

\subsection{Supply Chain Security}

The social engineering and the pressure on the unpaid volunteer maintaining
the \text{xz-utils} project started a discussion around the security and
resilience of such packages and the reasonableness of creating software
depending on said projects. Several open source packages are used as
dependencies by large corporations, as maintainers of the famous
\texttt{ffmpeg} media toolkit state: ''\textit{The xz fiasco has shown how a
    dependence on unpaid volunteers can cause major problems. Trillion
dollar corporations expect free and urgent support from volunteers.}``
\cite{twitter2024ffmpeg}. According to \cite{twitter2024ffmpeg} Microsoft is
not interested in a support contract for long term maintenance and
''\textit{[...] offered a one-time payment of a few thousand dollars
instead...investments in maintenance and sustainability are unsexy [...]}``.

Open source projects are the prime candidate for threat actors to inject
malware into, due to the fact that the maintainers are volunteers, unpaid
and most often not supported by companies and organisations who are directly
benefiting from the software volunteers create, maintain and extend. 
